\section{Functions}

\begin{df}
	Let $A,B$ be sets. Let $C$ be a subset of $A \times B$ such that 
	\begin{align*}
	\forall a \in A ~ \exists_1 b \in B: \qquad (a,b) \in C
	\end{align*} 
	Then a \emph{function} $F$ from $A$ to $B$ is defined as the triple:
	\begin{align*}
		F = (A,B,C)
	\end{align*}
	If $(a,b) \in C$ we write $b=F(a)$ and say that $b$ is the \emph{value} of $F$ at $a$.
	$A$ is called the \emph{domain} of $f$. $B$ is called the \emph{range} of $f$.
	\begin{align*}
	\{b \in B| \quad \exists a \in A:~ f(a) = b \} \subseteq B
	\end{align*}
	is called the \emph{image} of $f$.
	
	Notation: $F:~ A \to B$ for a function $F$ from $A$ to $B$.
\end{df}
A function is a set of arrows. For a set of arrows to be a function means that for all $a \in A$ there is a unique arrow starting at $a$.

\begin{rk}
	In the following, if we talk about a function $F=(A,B,C)$ we will mostly refer to the set $C$.
\end{rk}
	
\begin{ex}
	\begin{align*}
	f& :  ~ \R  \to \R: \qquad f(x) = x^2 \\
	f & = \{(x,y) \in \R \times \R| \quad y = x^2 \} \\
	f'& : ~\R  \to [0, \infty): \qquad f'(x) = x^2\\
	f' & = \{(x,y) \in \R \times [0, \infty)| \quad y = x^2 \}
	\end{align*}
	Note that $f \neq f'$:	
	\begin{align*}
	\range(f) & = \R \\
	\image(f) & = \R_{\ge 0} \\
	\range(f') & = \R_{\ge 0} \\
	\image(f') & = \R_{\ge 0}
	\end{align*}
\end{ex}

\begin{rk}
	This is true due to the fact that for every $y \in \R_{\ge 0}$ there exists an $x \in \R$ with $x^2 = y$ which will be proven later.
\end{rk}

\begin{df}
	A function $f$ is \emph{injective} if 
	\begin{align*}
	\forall a_1, a_2 \in A: \qquad f(a_1) = f(a_2) \quad \Rightarrow \quad a_1 = a_2 
	\end{align*}
\end{df}
No two elements of $A$ are mapped to the same element of $B$.

\begin{df}
	A function $f$ is \emph{surjective} if 
	\begin{align*}
	\image(A) = \range(A)
	\end{align*}
\end{df}
For all $b \in B$ there is some $a \in A$ that is mapped to $b$ by $f$.

\begin{df}
	A function $f$ is \emph{bijective} if it is both injective and surjective.
\end{df}

\begin{df}
	Let $A$ be a set. Define 
	\begin{align*}
		\id_A = \{(a,a)| ~ a \in A\}
	\end{align*} That means for all $a \in A, \id_A(a) = a$.
\end{df}

\begin{df}
	Suppose $f: A \to B$ is a function and $f: B \to C$ is a function. Then we can form a function:
	\begin{align*}
	h & = g \circ f: \quad A \to C \\
	h(a) & = g(f(a))
	\end{align*}
	The function $f$ is \emph{invertible} if
	\begin{align*}
	\exists g: B \to A: \qquad g \circ f = \id_A \quad \wedge \quad f \circ g = \id_B	
	\end{align*}
	
	Notation:
	$g = f^{-1}(x)$ is called the \emph{inverse} of $f$.
\end{df}
$f$ is invertible means that if you reverse the arrows of $f$, then that is a function.

\begin{ex}
	\begin{itemize}
		\item The function
	\begin{align*}
	& g:  [0, \infty) \to [0, \infty) \\
	& g(x) = x^2
	\end{align*}
	is invertible and the inverse is the function
	\begin{align*}
	& g^{-1}: ~ [0,\infty) \to [0, \infty) \\
	& g^{-1}(y)  = \sqrt y
	\end{align*}
	\item
	Let us think of 
	\begin{align*}
	& f': \R \to [0, \infty) \\
	& f'(x) = x^2
	\end{align*}
	Look at 
	\begin{align*}
	g
	f^{-1}: & ~ [0,\infty) \to [0, \infty) \\
	f^{-1}(y) & = \sqrt y \\
	f' \circ f^{-1}(y) & = (\sqrt y ) ^2 = y \\
	\Rightarrow \qquad f' \circ f^{-1} & = \id_{[0,\infty)}
	\end{align*}
	but
	\begin{align*}
	f^{-1} \circ f'(x) & = \sqrt{x^2} = |x| \\
	\Rightarrow \qquad f^{-1} \circ f' & \neq \id_\R
	\end{align*}
	
	\begin{rk}
		We did not prove that $f'$ is not invertible. We just showed that $g$ is not its inverse.
	\end{rk}	
\end{itemize}
\end{ex}

\begin{pp}
	$f$ is invertible iff $f$ is bijective.
\end{pp}

\begin{proof}
	$\Rightarrow$ \\
	Suppose that $f$ is invertible. We have to show that $f$ is injective and surjective. That means
	\begin{align*}
	\exists g: B \to A: \qquad g \circ f & = \id_A \tag{1} \\
	f \circ g & = \id_B \tag{2}
	\end{align*}
	$f$ is injective:
	\begin{align*}
	f(a_1) = f(a_2) \quad \overset{(1)}{\Rightarrow} \quad g \circ f(a_1) = a_1 = f \circ f(a_2) = a_2
	\end{align*}
	$f$ is surjective. By (2) for all $b \in B$, $b = f(g(b))$. So indeed there exists $a \in A$ such that $b= f(a)$, namely $a=g(b)$. \\
	$\Leftarrow$ \\
	Suppose that $f$ is injective and surjective. Take
	\begin{align*}
		g = \{(b,a) \in B \times A| \quad (a,b) \in f \} \subseteq B \times A
	\end{align*}
	$f$ is injective and surjective means precisely that $g$ is a function.
	
	It is obvious that $g \circ f = \id_A$ and $f \circ g = \id_B$. 
\end{proof}


\begin{tm} \emph{Fermat's little theorem.} \\
	If $p$ is prime and $p \nmid a$ then $a^{p-1} \equiv 1 \mod p$.
\end{tm}

\begin{proof}
	%($a \in \Z/p\Z^x$ is a group with $p-1$ elements. Therefore the order of any element divides $p-1$.)
	Consider the function
	\begin{align*}
	f& : \Z/p\Z^x \to \Z/p\Z^x \\
	f& : x \mapsto ax
	\end{align*}
	This is an invertible function. We know that $a$ has a multiplicative inverse $b \in (\Z/p\Z^x)$. Take 
	\begin{align*}
	g & : \Z/p\Z^x \to \Z/p\Z^x \\
	g & : x \mapsto bx
	\end{align*} 
	Then $g = f^{-1}$ because $ab \equiv ba \equiv 1 \mod p$.
	\begin{align*}
	\left\{ \overline a, \overline{2a}, \dots, \overline {(p-1)a} \right\}&  = \{1,2, \dots p-1\} \\
	\Rightarrow \qquad \qquad a^{p-1} (p-1)! & \equiv (p-1)! \mod p
	\end{align*}
	Dividing through by $(p-1)!$ gives us
	\begin{align*}
	a^{p-1} \equiv 1 \mod p
	\end{align*}
\end{proof}

\begin{tm}
	\emph{Chinese Remainder Theorem} \\
	Suppose $\hcf(n,m)=1$. Then for all $a,b$ the equation
	\begin{align*}
	\left.
	\begin{array}{ll}
	x & \equiv a \mod n \\
	x & \equiv b \mod m
	\end{array} \right\} \tag{$*$} 
	\end{align*}
	has a solution $x \in \Z$. In fact, $x$ is unique $\mod m n'$.
\end{tm}

\begin{proof}
	\begin{align*}
	x = a + pn & = b + qm\ \\
	\Rightarrow \qquad pn - qm & = b-a
	\end{align*}	
	$\hcf(n,m) = 1$ implies that $p$ and $q$ exist. \\
	As for uniqueness $\mod nm$, suppose that $x_0, x_1$ are two solutions of $(*)$. Then
	\begin{align*}
	& & x_0 - x_1 & \equiv 0 \mod n \\
	& & & \equiv 0 \mod m \\
	& \Rightarrow & \qquad n,m & | x_0 - x_1 \tag{1}\\
	 (1) \wedge \hcf(n,m) = 1 & \Rightarrow &  nm & | x_0-x_1
	\end{align*}
\end{proof}


	\begin{pp}
		Euler's function $\varphi$ is multiplicative, that is:
		If $\hcf(n,m)=1$ then $\varphi(nm) = \varphi(n) \varphi(m)$.
	\end{pp} 

\begin{proof}
	The Chinese remainder theorem says the following:
	We can define a function $f$ with
	\begin{align*}
	f&: \Z/n\Z \times \Z/m\Z \to \Z/nm\Z \\
	f&: (a,b) \mapsto  \text{solution of $(*)$}
	\end{align*}
	and $f$ is invertible. In fact, $f$ is the inverse of:
	\begin{align*}
	g& : \Z/nm\Z \to  \Z/n\Z \times \Z/m\Z\\
	g& : [c] \mod mn \mapsto \left([c] \mod n, [c] \mod m \right)
	\end{align*}
	This shows that 
	\begin{align*}
	f': (\Z/n\Z)^x \times (\Z/m\Z)^x \to (\Z/nm\Z)^x 
	\end{align*}
	is invertible. Hence, the cardinality of both sets is the same.
\end{proof}

\begin{pp}
	For all $n \in \N$
	\begin{align*}
	\varphi(n) = n \prod_{p ~\text{prime}, ~p | n} \left(1- \frac 1 p \right)
	\end{align*}
\end{pp}

\begin{proof}
Suppose that $p$ is prime and that $k=p^a$.
\begin{align*}
\varphi (k) & = |\{c \in \{1,2, \dots, p-1\}| ~ p \nmid c \} | \\
& = p^a - p^{a-1} \tag{1}
\end{align*}
Therefore, by multiplicativity of $\varphi$, if 
\begin{align*}
n = p_1^{a_1} p_2^{a_2} \dots p_r^{a_r} 
\end{align*}
then
\begin{align*}
	\varphi(n) & = \varphi \left(p_1^{a_1} \right) \varphi \left( p_2^{a_2} \right) \dots \varphi \left( p_r^{a_r} \right) \\
	& \overset{(1)}{=} \left( p_1^{a_1} p_1^{a_1-1} \right) \left(p_2^{a_2} - p_2^{a_2-1} \right) \dots \left(p_r^{a_r} - p_r^{a_r-1} \right) \\
	& = p_1^{a_1} p_2^{a_2} \dots p_r^{a_r} \left( 1 - \frac 1 {p_1} \right) \left( 1 - \frac 1 {p_2} \right) \dots \left( 1 - \frac 1 {p_r} \right) \\
	& = n \left( 1 - \frac 1 {p_1} \right) \left(1 - \frac 1 {p_2} \right) \dots \left( 1 - \frac 1 {p_r} \right)  .
\end{align*}
\end{proof}


\begin{ex}
	\begin{align*}
	\varphi(9) & = 9 - 3 \\
	& = \{ c \in \{ 1,2, \dots, 9 \} | ~ 3 \nmid c \}
	\end{align*}
\end{ex}
Consider the equation $x^k \equiv a \mod m$
We will discuss 2 cases:
\begin{enumerate}
	\item
	$m=p$ prime 
	\item
	$m = pq$ product of two distinct primes
\end{enumerate}

\begin{pp}
	Suppose $\hcf(a, p-1) = 1$. Then
	\begin{align*}
	x^k \equiv a \mod p  \tag{$*$}
	\end{align*}
	has a unique solution modulo $p$.
\end{pp}

\begin{proof}
	First, there exist $u,v \in \N$ such that 
	\begin{align*}
	ku - (p-1)v = 1.
	\end{align*}
	then $x = a^u$ is a solution of $(*)$. Indeed, 
	\begin{align*}
	x^k & = a^{ku} = a^{1+(p-1)v} \\
	& = a \left( a^{p-1} \right)^v \\
	& \equiv a \mod p.
	\end{align*}
	By Fermat's little theorem. Conversely, suppose that $x^k \equiv a \mod p$. Then by Fermat's little theorem,
	\begin{align*}
	x & \equiv x x^{(p-1) v} = x^{1+(p-1)v} \mod p\\
	& = x^{ku} \\
	& \equiv a^u \mod p
	\end{align*}
	We have shown that $x= a^u$ is a unique solution of $(*)$ mod $p$.
\end{proof}

\begin{pp}
	Suppose $p$ and $q$ are distinct primes $\hcf(a, pq) = 1$. Then
	\begin{align*}
	a^{(p-1)(q-1)} \equiv 1 \mod pq.
	\end{align*}
\end{pp}

\begin{proof}
	By Fermat's little theorem
	\begin{align*}
	a^{(p-1)(q-1)} &\equiv 1^{q-1} \equiv 1 \mod p \\
	a^{(p-1)(q-1)} & \equiv 1 \mod q.
	\end{align*}
	And
	\begin{align*}
	x \equiv 1 \mod p \\
	x \equiv 1 \mod q
	\end{align*}
	has a unique solution by the Chinese remainder theorem. Therefore, $x=1$ is the unique solution modulo $pq$.
	Another solution is $a^{(p-1)(q-1)}$. So $a^{(p-1)(q-1)} \equiv 1 \mod pq$.
\end{proof}

\begin{cl}
	Suppose that $p,q$ are distinct primes. Then for all $a$, for all $v \in \N$,
	\begin{align*}
	a^{1+v(p-1)(q-1)} \equiv a \mod pq.
	\end{align*}
\end{cl}

\begin{proof}
	\begin{align*}
	a^{1+v(p-1)(q-1)} & \equiv a \mod p \\
	a^{1+v(p-1)(q-1)} & \equiv a \mod q
	\end{align*}
	Because of Proposition 4.7. If $p |a$ or $q|a$ these statements obviously hold as well. By the uniqueness part of the Chinese remainder theorem 
	\begin{align*}
	a^{1+v(p-1)(q-1)} & \equiv a \mod pq.
	\end{align*}
\end{proof}

\begin{pp}
	Suppose $\hcf(k,(p-1)(q-1)) = 1$. Then
	\begin{align*}
	x^k \equiv a \mod pq \tag{$*$}
	\end{align*}
	has a unique solution modulo $pq$.
\end{pp}

\begin{proof}
	There exist $u,v \in \N$ such that
	\begin{align*}
	ka-(p-1)(q-1) v = 1.
	\end{align*}
	Therefore, $x=a^u$ is a solution of $(*)$:
	\begin{align*}
	x^u & \equiv a^{vu}  \mod pq\\
	& = a a^{(p-1)(q-1) r} \\
	& \equiv a \mod pq 
	\end{align*}
	By Proposition 4.7. Conversely, suppose $x^k \equiv a \mod pq$:
	\begin{align*}
	x & \equiv x x^{(p-1)(q-1)r} \mod pq \\
	& = x^{ku} \\
	& \equiv a^u \mod pq
	\end{align*}
	Thus, $x=a^k$ is the unique solution of $(*)$ mod $pq$.
\end{proof}

\begin{ex}
	Solve for $x$ :
	\begin{align*}
	x^{53} \equiv -38 \mod 119
	\end{align*}
	We check
	\begin{align*}
	119 = 7 \cdot 17 \\
	\hcf(38,119) = 1 \\
	\hcf(53, 96) = 1.
	\end{align*}
	Hence, we can apply Proposition 4.8 to this.
	To find $x$ we need to find $n,r \in \N$ such that
	\begin{align*}
	53 u - 96 r = 1.
	\end{align*}
	Euklid's Algorithm gives us
	\begin{align*}
	5 \cdot 29 - 96 \cdot 1601 .
	\end{align*}
	By Proposition 4.8, $x = (-38)^{29} \mod 119$. To calculate this we use the method of successive squares:
	\begin{align*}
	& & 29 & = 16 +8+4+1 \\
	& & (-38)^2 & = 1444 \equiv 16 \mod 119 \\
	& \Rightarrow & (-38)^4 & = 16^2 \equiv 18 \mod 119 \\
	& \Rightarrow & (-38)^8 & = 18^2 \equiv -33 \mod 119 \\
	& \Rightarrow & (-38)^{16} & = (-33)^2 \equiv 18 \mod 119 
	\end{align*}
	Therefore,
	\begin{align*}
	(-38)^{29} = 18 \cdot (-33) \cdot 18 \cdot (-38) \equiv 30 \mod 119.
	\end{align*}
	The unique solution is 
	\begin{align*}
	x \equiv 30 \mod 119.
	\end{align*}
\end{ex}



\subsection{RSA encryption}

A real-life application of simple mathematical ideas is RSA. It is called "public key cryptography". We publish the means to encrypt messages sent to us. Nevertheless, only we have the information that allows us do decrypt the messages.

Here is what we do:
\begin{center}
\begin{tabularx}{.5\textwidth}{XX}
	\toprule
	secret & public \\
	\toprule
	We choose two large primes $p$ and $q$. & $N = pq$ \\
	\midrule
	We choose $e$ such that $\hcf e,(p-1(q-1) = 1$. & e \\
	\bottomrule
\end{tabularx}
\end{center}
A message is an element $a \in \Z/n\Z$. If now an arbitrary person, let us call this person Assange, wants to send an encrypted message $a$ to us, he has to compute
\begin{align*}
b = a^e \in \Z/N\Z .
\end{align*}
He keeps $a$ to himself and sends $b$.
In order to decrypt, we do the following:
Find $d$ such that
\begin{align*}
de \equiv 1 \mod (p-1)(q-1).
\end{align*}
Then $a \equiv b^d \mod N$. This works by the Proposition 4.8.

\begin{ex} \mbox \\
	\begin{center}
		\begin{tabularx}{.5\textwidth}{XX}
			\toprule
			secret & public \\
			\toprule
			$p=7,~ q = 17$ & $N = 119$ \\
			\midrule
			$(p-1)(q-1) = 96$ and $e=53$ such that $\hcf(53,96) = 1$ & $e=53$ \\
			\bottomrule
		\end{tabularx}
	\end{center}
	\emph{Encryption.}
	Assange wants to send us the message $a=30$ (don't tell anybody).
	He sends us the encrypted message:
	\begin{align*}
	& & b & = 30^{53} \mod 119 \\
	& & 53 & = 32+16+4+1 \\
	& & 30^2 & = 900 \equiv 67 \mod 119 \\
	& \Rightarrow & 30^4 & = 67^2 = 4489 \equiv -33 \mod 119 \\ 
	& \Rightarrow & 30^8 & = (-33)^2 = 1089 \equiv 18 \mod 119 \\
	& \Rightarrow & 30^{16} & = 18^2 = 324 \equiv -33 \equiv \mod 119 \\
	& \Rightarrow & 30^{32} & = (-33)^2 \equiv 18 \mod 119 \\
	& & 30^{53} & \equiv 30 \cdot (-33) \cdot (-33) \cdot 18 \mod 119 \\
	& & & \equiv  -38 \mod 119
	\end{align*}
	He sends us the message -38.
	
	\emph{Decryption.}
	Find $d$ such that $53d \equiv 1 \mod 96$. Solve for $u,v \in \N$:
	\begin{align*}
	53 u - 96 = 1 
	\end{align*}
	To solve this we use Euklid's algorithm:
	\begin{align*}
	53 \cdot 29 - 96 \cdot 16 = 1.
	\end{align*}
	So $d = 29$. To decrypt we compute 
	\begin{align*}
	(-38)^{29} \equiv 30 \mod 119 
	\end{align*}
\end{ex}

\subsection{Basic counting techniques}


\begin{pp}
	Let $S$ be a finite set. Define
	\begin{align*}
	P(S) : = \{ a|~ a \subseteq S \}.
	\end{align*}
	Then 
	\begin{align*}
	|P(S)| = 2^{|S|}.
	\end{align*}
\end{pp}
	\begin{proof}
		Let $A,B$ be sets. Write
		\begin{align*}
		A^B = \{ f: B \to A| ~ f~ \text{a function} \}
		\end{align*}
		If $A,B$ are finite then:
		\begin{align*}
		\left| A^B \right| = |A|^{|B|}
		\end{align*}
		Let $\underline 2 = \{0,1\}$. There is a canonical (i.e. naturally defined) bijection:
		\begin{align*}
		f: P(S) \to \left(\underline{2}^S = \{f:~ S \to \underline 2 \} \right)
		\end{align*}
		This bijection can be defined as follows: Suppose that $A \in P(S)$; that is $a \subseteq S$.
		\begin{align*}
		f: A \mapsto \left( K_A: ~ S \to \{0,1\} : \quad  K_A(s) = 
		\left\{
		\begin{array}{l l}
		1, & s \in A \\
		0, & s \not\in A
		\end{array}\right. \quad  \right)
		\end{align*}
		$f$ is a bijection. In fact, its inverse is
		\begin{align*}
		|P(S)| = |2^S| = 2^{|S|}.
		\end{align*}
	\end{proof}



\begin{pp}
	The number of ways of ordering the numbers $1,2, \dots ,n$ is $n!$.
\end{pp}

\begin{pp}
	The number of subsets of order $r$ of $\{1,2, \dots, n\}$ is
	\begin{align*}
	 \binom n r = \frac{n!}{r!(n-r)!}.
	\end{align*}
\end{pp}

\begin{proof}
	We can order $1,2, \dots,n$ by choosing $r$ elements first, ordering these elements and the others.
	\begin{align*}
	n! = \binom n r r! (n-r)!
	\end{align*}
	The proposition follows.
\end{proof}

\begin{tm}
	\begin{align*}
	(x+y)^n = \sum_{k=0}^n \binom n k x^k y^{n-k}
	\end{align*}
\end{tm}

\begin{proof}
	Write
	\begin{align*}
	(x+y)^n = \underbrace{(x+y) (x+y) \dots (x+y)}_{n ~\text{times}}
	\end{align*}
	You score the term $x^{n+r} y^r$ by choosing $r$ $y$s out of the $n$ brackets.
\end{proof}

\subsubsection{Partitions and multinomial coefficients}

\begin{df}
	A partition of a set $S$ is a collection of subsets of $S$:
	\begin{align*}
	\forall a \in A, S_a \subseteq S: \qquad a\neq b \Rightarrow S_a \cap S_b = \emptyset \qquad \wedge \qquad S = \bigcup_{a\in A} S_a
	\end{align*}
\end{df}

\begin{rk}
	A typical example of this is where $A$ is the set of equivalence classes of an equivalence relation. In fact, all examples of partitions arise in this fashion.
\end{rk}

\begin{df}
An ordered partition is a partition where the sets $S_a$ are with a chosen order.
\end{df}

\begin{ex}
	\begin{align*}
	S & = \{1,2,3,4,5,6,7,8 \} \\
	\{\{1,2,3,4\}, \{5,6 \}, \{7,8 \}\} & = \{\{1,2,3,4\}, \{7,8 \}, \{5,6\}\}
	\end{align*}
	but
	\begin{align*}
	(\{1,2,3,4\}, \{5,6 \}, \{7,8 \}) \neq (\{1,2,3,4\}, \{7,8 \}, \{5,6\}).
	\end{align*}
\end{ex}

\begin{pp}
	The number of ordered partitions of $\{1,2, \dots,n\}$ into $k$ subsets $S_1, S_2, \dots S_k$ of orders $r_1, r_2, \dots r_k$ is
	\begin{align*}
	\binom n {r_1, r_2, \dots, r_k} = \frac n {r_1! r_2! \dots, r_k!}
	\end{align*}
	where $r_1+r_2+ \dots + r_k = n$.
\end{pp}

\begin{proof}
	Order $1,2, \dots,n$ by choosing $r_1, \dots, r_k$ and ordering the first $r_1$ then $r_2$ to $r_k$. We get:
	\begin{align*}
	n!= \binom n  {r_1, r_2, \dots, r_k} r_1! r_2! \dots, r_k!
	\end{align*}
	The proposition follows.
\end{proof}

\begin{rk}
	$\displaystyle \binom n {r,n-r} = \binom n r$
\end{rk}

\begin{tm}
	\emph{Multinomial theorem} \\
	\begin{align*}
	(x_1+ x_2+ \dots + x_k)^n = \sum_{r_1, \dots, r_k \ge 0, ~r_1 +\dots+ r_n = n} \binom n {r_1, \dots, r_n} x_1^{r_1} \dots x_k ^{r_k}
	\end{align*}
\end{tm}

\begin{proof}
	Write 
	\begin{align*}
	(x_1+ x_2+ \dots + x_k)^n = \underbrace{(x_1+ x_2+ \dots + x_k) \dots (x_1+ x_2+ \dots + x_k)}_{n~\text{times}}
	\end{align*}
	We score the term $x_1^{r_1} \dots x_k ^{r_k}$ by choosing $r_1$ $x_1$s, $r_2$ $x_2$s up to $r_k$ $x_k$s out of the $n$ brackets.
\end{proof}

\begin{ex}
	Find the "constant" coefficient in the expression of
	\begin{align*}
	\left(x+y+z + \frac 1 {xyz} \right)^n
	\end{align*}
	Expand with the multinomial theorem:
	\begin{align*}
	\sum_{r_1+r_2+r_3+r_4 = n} \binom n {r_1, r_2, r_3, r_4} \frac{x^{r_1} y^{r_2} z^{r_3}}{(xyz)^r4}
	\end{align*}
	The fraction is a constant if
	\begin{align*}
	r_1=r_2=r_3=r_4=k \quad \wedge \quad n = 4k
	\end{align*}
	The answer is:
	\begin{align*}
	c_n = \left\{ \begin{array}{ll}
	0, &  4 \nmid n \\
	\binom{4k}{k,k,k,k} = \frac{(4k)!}{(k!)^4}, &  n = 4k
	\end{array}
	\right.
	\end{align*}
\end{ex}


\begin{pp}
	\emph{Inclusion-Exclusion principle} \\
	Let $A_1, A_2, \dots, A_n$ be finite sets. Then
	\begin{align*}
	\left| \bigcap_{k=1}^n A_k \right| = \sum_{1 \le i_1 < i_2 < \dots < i_r \le n} (-1)^{r-1} \left| A_{i_1} \cap \dots \cap A_{i_r} \right|.
	\end{align*}
\end{pp}


\begin{ex}
	Let 
	\begin{align*}
	S = \{ k \in \N |~ 0 \le k < 360, ~\hcf(k,360) = 1 \}
	\end{align*}
	On the one hand, we know $|S| = \varphi(360)$ and we have a formula for this. Hence, we compute $|S|$ as an application of the inclusion-exclusion formula:
	\begin{align*}
	360 = 2^3 \cdot 3^2 \cdot 5
	\end{align*}
	Define
	\begin{align*}
	\Omega & = \{k \in \N|~ 0 \le k < 360 \} \\
	A_2 & = \{ k \in \Omega| ~ 2 |k \} \\
	A_3 & = \{ k \in \Omega| ~ 3 |k \} \\
	A_5 & = \{ k \in \Omega| ~ 5 |k \} .
	\end{align*}
	Then
	\begin{align*}
	S & = \Omega \backslash (A_2 \cup A_2 \cup A_5) \\
	|S| & = |\Omega| - |A_2 \cup A_3 \cup A_5|
	\end{align*}
	If $d|360$ write
	\begin{align*}
	A_d & = \{ k \in \Omega| ~ d |k \} .
	\end{align*}
	Then
	\begin{align*}
	A_2 \cup A_3 & = A_6 \\
	A_2 \cup A_5 & = A_{10} \\
	A_3 \cup A_3 & = A_{15} \\
	A_2 \cup A_3 \cup A_5 & = A_{30} 	
	\end{align*}
	The inclusion-exclusion principle gives us
	\begin{align*}
	|S| & = |\Omega| - |A_2| - |A_3| - |A_5| + |A_6| + |A_{10}| + |A_{15}| - |A_{30}| \\
	& = 360 - 180 - 120 - 72 + 60 + 36 + 24 - 12 \\
	& = 96
	\end{align*}
\end{ex}