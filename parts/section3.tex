\section{Natural Numbers}
$\mathbb N$ is defined by three axioms:
\begin{ax}
	$ 0 = \emptyset \in \mathbb N$ 
\end{ax}
\begin{ax}
	If $n \in \mathbb N$ then 
	\begin{align*}
	n+1 \overset{def} = n \cup \{n\} \in \mathbb N.
	\end{align*}
\end{ax}
\begin{ax} \emph{Smallest element axiom.} \\
Let $\emptyset \neq S \subseteq \mathbb N$. Then $S$ has a smallest element, i.e.
\begin{align*}
\exists k: ~\forall a \in S \qquad k \le a.
\end{align*}
\end{ax}

\begin{rk}
	A smallest element is clearly unique.
\end{rk}

\begin{ex}
	\begin{align*}
	1 & = 0+1 = \emptyset \cup \{ \emptyset \} = \{ \emptyset \} \\
	2 & = 1+1 = \{ \emptyset \} \cup \{ \{ \emptyset \} \} = \{ \emptyset, \{ \emptyset \} \} 
	\end{align*}
\end{ex}

\begin{rk}
	Although this definition may seem simple and wonderful, we will not bother with the first two axioms in the further discussion of $\N$ and assume common properties of $+, -, \cdot$, and $:$ (which we are not going to define rigorously).
\end{rk}


\begin{tm}
\begin{align*}
\forall n,p \in \mathbb N,~ \exists_1 q,r \in \mathbb N, ~0\leq r <p: \qquad n=pq+r.
\end{align*}
\end{tm}

\emph{Special case.}

For $p=2$ this says that there exists a $q$ such that either $n=2q$ or $n=2q+1$ but not both.


\begin{proof} Let
\begin{align*}
S = \{ y \in \mathbb N | ~ \exists k \in \N: ~ y = n-pk \} .
\end{align*}
$S \neq \emptyset$ because $n \in S$ for $k=0$.
Axiom 3.3 says that $S$ has a smallest element. Let this smallest element be $y_0 = n-p k_0$. We claim $0 \leq y_0 <p$. 

Assume for contradiction $y_0 \geq p$. Then we can construct $y_0'$ with
\begin{align*}
y'_0 & = y-p \\
& = (n-pk_0) -p \\
& = n-p(k_0+1) \in S
\end{align*}
Due to our definition $y_0'<y$. Hence $y_0$ can not be the smallest element of $S$ and we get a contradiction.

Now that we know that $0 \le y_0 < p$ and obviously $n=y_0+pk_0$, we can choose $r=y$ and $q = k_0$. 
It remains to show uniqueness.

To show uniqueness suppose there exists $q_1, q_2, r_1,r_2 \in \N, 0 \le r_1,r_2 <p$. \hfill (1) \\ 
Without loss of generality we may assume $r_1 \leq r_2$. \hfill (2) \\
Hence
\begin{align*}
n & = pq_1 +r_1 \tag {3.1} \\
n & = pq_2 +r_2. \tag {3.2}
\end{align*}
 Subtracting (3.2) from (3.1) gives us
\begin{align*}
0 & = p(q_1-q_2) + r_1 -r_2 \\
\Leftrightarrow \qquad r_2-r_1 & = p(q_1-q_2)
\end{align*}
Using (1) and (2) we get
\begin{align*}
0 \leq r_2-r_1 = (q_1-q_2) p <p.
\end{align*}
So $0\leq q_1-q_2 <1$ and therefore $q_1=q_2$ and $r_1=r_2$.
\end{proof}



\subsection{Proof by Induction}
\begin{tm}
	The \emph{Principle of Induction} is the following: \\
	Suppose that $\forall n \in \N$ we are given a statement $P_n$. We assume that:
	\begin{enumerate}
		\item 
		$P_0$ holds.
		\item
		$\forall n \in \N, (Pn \Rightarrow P_{n+1})$ holds.
	\end{enumerate}
	Then for all  $n \in \N$, $P_n$ holds.
\end{tm}
\begin{ex} Let
\begin{align*}
 P_n: \qquad 0+1+2+\dots +n = \frac{n(n+1)} 2.
\end{align*}
Let us show $P_n \Rightarrow P_{n+1}$.
Assume that 
\begin{align*}
0+1+2+\dots+n = \frac{n(n+1)} 2
\end{align*}
then
\begin{align*}
0+1+2+\dots+n +(n+1) = (0+1+2+\dots+n) +(n+1)
= \frac{n(n+1)} 2 + n+1 = \frac{(n+1)(n+2)}{2}
\end{align*}
$P_0$ is the statement that $0=0$. Therefore $\forall n$ the formula is true.
\end{ex}
\begin{proof}
We argue by contradiction. Suppose that the conclusion is false. That means:
\begin{align*}
& \exists n \in \N: & \overline{P_n}
\end{align*}
In other words:
\begin{align*}
S= \{ n \in \N| ~ \overline{P_n} \} \neq \emptyset
\end{align*}
Let $k$ be the smallest element of $S$. $k$ exists by the smallest element axiom.
$k-1 < k$, therefore $k-1 \in S$, thus $P_{k-1}$ holds. But:  
\begin{align*}
P_{k-1} \Rightarrow P_k
\end{align*}
\end{proof}
\begin{ex}
The Fibonacci sequence.
$\forall n \in \N$ define $F_n$ inductively by the formula:
\begin{align*}
F_0 = 0, \quad F_1 = 1, \quad \forall n \geq 2 F_n = F_{n+1} +F_{n+2}
\end{align*}
Let us prove by induction that:
\begin{align*}
F_n = \frac {1}{\sqrt 5} \left( \left( \frac{1+\sqrt 5} 2 \right)^n - \left( \frac{1-\sqrt 5}{2} \right)^n \right)   = \frac{\varphi^n-\psi^n}{\sqrt 5}
\end{align*}
\end{ex}
The really interesting thing would be to understand how one can "come up" with a formula like this. 
Another interesting thing would be to "stare" at the formula and see what we can learn from it about life.
Instead wo focus on a "minor" print of logic.

\emph{Wrong proof.} To prove by induction you need to declare at the outset, $\forall n$ what is $P_n$. Your instinct here will be to say 
\begin{align*}
& P_n: & F_n = \frac {1}{\sqrt 5} \left( \left( \frac{1+\sqrt 5} 2 \right)^n - \left( \frac{1-\sqrt 5}{2} \right)^n \right) 
\end{align*}
Then you will write:
\begin{align*}
F_{n+1} & = F_n + F_{n+1} \\
 & = ( \dots ) + ( \dots)
\end{align*}
\begin{rk}
You have used both, $P_{n-1}$ and $P_n$. However, for induction you can only use $P_n$. 
\end{rk}
\begin{proof}
We use the principle of induction with:
\begin{align*}
Q_n = (P_n \wedge P_{n+1} )
\end{align*}
We need to show $\forall n: ~Q_n \Rightarrow Q_{n+1}$. Suppose $(P_n \wedge P_{n+1} \Rightarrow P_{n+1}) \Rightarrow ((P_n \wedge P_{n+1}) \wedge (P_{n+1} \wedge P_{n+2}))$.
Hence we only need to proof that $P_n \wedge P_{n+1} \Rightarrow P_{n+2}$
Assume $P_n \wedge P_{n+1}$, then:
\begin{align*}
F_{n+1} = F_{n+1} +F_n = & \frac 1 {\sqrt 5} \left( \varphi^{n+1} - \psi^{n+1} \right) + \frac 1 {\sqrt 5} \left( \varphi^n -\psi^n \right)  \\
= & \frac 1 {\sqrt 5} \varphi^n (\varphi+1) + \frac 1 {\sqrt 5} \psi^n(\psi+1) 
\end{align*}
Since $\varphi$ and $\psi$ are solutions of the equation $x^2-x-1=0$ we can rewrite that as:
\begin{align*}
 & \frac 1 {\sqrt 5} \varphi^n \varphi^2 + \frac 1 {\sqrt 5} \psi^n \psi^2 \\
= & \frac {1}{\sqrt 5} \varphi^{n+2} + \frac 1 {\sqrt 5} \psi^{n+2}
\end{align*}
So $P_{n+2}$ holds. We have shown that $\forall n: ~ Q_n \Rightarrow Q_{n+1}$. To finish the proof we need $Q_0 = (P_0 \wedge P_1)$.
\begin{align*}
& P_0 & F_0 &= \frac 1 {\sqrt 5} \left( \varphi^0 - \psi^0 \right) = 1 \\
& P_1 & F_1 & = \frac 1 {\sqrt 5} \left( \varphi^1 - \psi^1 \right) =1
\end{align*}
\end{proof}
\begin{tm}
Principle of strong induction. \\
Suppose that $\forall n \in \N$ we are given a statement $Q_n$.
Assume that:
\begin{enumerate}
\item
$Q_0$ holds;
\item
$\forall n, ~ (\forall k \leq n: ~ Q_k) \Rightarrow Q_{n+1}$
\end{enumerate}
Then $\forall n \in \N, ~Q_n$ holds.
\end{tm}

\begin{proof}
Apply induction with:
\begin{align*}
(Q_0 \wedge Q_1 \wedge \dots \wedge Q_n)
\end{align*}
\end{proof}


\subsection{Prime numbers}
\begin{df}
$n \neq 0,1 \in \N$ is irreducible if:
\begin{align*}
\forall u,v \in \N: \quad n = uv \quad \Rightarrow \quad u=1 \vee v =1
\end{align*}
\end{df}
\begin{tm}
Every $n \in \N, n \neq 0,1$ is the product of irreducibles.
\end{tm}
\begin{proof}
We are going to prove the statement by strong induction.
Let $Q_n$ be the statement that $n$ is the product of irreducibles. 

$Q_0$ clearly holds. 

Assume $Q_n$ for $k \leq n$.
If $n+1$ is irreducible then $Q_{n+1}$.
Otherwise $n+1= u \cdot v$ where $1 < u < n+1$ and $1<v<n+1$.
By $Q_u, u$ is prod of irreducibles.
By $Q_v, v$ is the product if irreducibles. Therefore, $Q_{n+1}$ holds.
\end{proof}


\begin{df}
For $c,a \in \mathbb Z$ we say that $c$ divides $A$ and write $c|A$ if 
\begin{align*}
\exists R \in \mathbb Z: \quad cR=A
\end{align*}
\end{df}
\begin{rk}
\begin{align*}
c|A_1 \wedge c|A_2 &\Rightarrow c|(A_1+A_2) \\
c|A & \Rightarrow \forall B \in \mathbb Z, c|AB
\end{align*}
\end{rk}
\begin{df}
\begin{align*}
\forall a,b \in \mathbb Z: \quad  hcf(a,b) = \text{Highest Common Factor} = \max\{t \in \mathbb Z| \quad t|a \wedge t|b \}
\end{align*}
\end{df}
\begin{rk}
\begin{align*}
\hcf(a,b) = \hcf(\pm a, \pm b) \in \N = \hcf(b,a)
\end{align*}
\end{rk}
Let us now consider the \emph{Division Algorithm} to compute the highest common factor. \\
Suppose $a,b \in \N$ with $a \ge b$. We know from last time:
\begin{align*}
\exists q,r \in \N, \quad 0 \le r < b: \quad a= bq+r
\end{align*}
Note that
\begin{align*}
(t|a \wedge t|b) \quad \Leftrightarrow \quad (t|b \wedge t|r)
\end{align*} This implies that $\hcf(a,b) = \hcf(b,r)$.
$a \ge b> r$ so the pair $(b,r)$ is smaller than the pair $(a,b)$, hence our algorithm will eventually come to an end. And I can assume by induction that I know to compute $k$ and $(b,r)$.


\begin{tm}
If $c = \hcf(a,b)$ then 
\begin{align*}
\exists y,x \in \Z: \quad c= ac+by
\end{align*}
\end{tm}
\begin{proof}
Assume $a \ge b > 0$ then write $a=bq+r$.
But what we said:
\begin{align*}
\hcf(a,b) =  \hcf(q,r) = c
\end{align*}
$(q,r)$ is smaller than $(a,b)$ so by induction there exist $x_0, g_0$ such that
\begin{align*}
c & = rx_0 + by_0 \\
& = (a-bq)x_0 + by_0 \\
& = ax_0 + b(y_0-qx_0)
\end{align*}
\end{proof}

\begin{ex}
Compute $\hcf(1734,371)=c$ and $x,y \in \Z$ such that $1734+371y=c$
\begin{align*}
& & 1734 & = 4\cdot 371 + 250 \\
& \Rightarrow & \hcf(1734,371) & = \hcf(371,250) \\
& & 371 & = 1 \cdot 250 +121 \\
& \Rightarrow & \hcf(371,250) & = \hcf(250,121) \\
& & 250 & = 2 \cdot 121 +8 \\
& \Rightarrow & \hcf(250,121) & = \hcf(121,8) \\
& & 121 & = 15 \cdot 8 +1 \\
& \Rightarrow & \hcf(121,8) & = \hcf(8,1)
\end{align*}
So $c=1$. 
\begin{align*}
1 & =-15 \cdot 8 +121 \\
& = -15 (-2 \cdot 121 +250 ) +121 \\
& =31 \cdot 121 - 12 \cdot 250 \\
& = 31 \cdot ( -1 \cdot 250 + 371) -15 \cdot 250 \\
&= -46 \cdot 250 +31\cdot 371 \\
& = -44(-4 \cdot 371 + 1734) +31 \cdot 371 \\
& = 215\cdot 371-46 \cdot 1734
\end{align*}
\end{ex}

\begin{df}
We say that $a,b \in \mathbb Z$ are \emph{co-prime} if $\hcf(a,c) = 1$.
\end{df}



\begin{df}
$p \in \N \backslash \{0,1\}$ is prime if:
\begin{align*}
\forall A,B \in \mathbb Z: \quad p|AB \Rightarrow p|A \vee p|Q
\end{align*}
\end{df}
\begin{tm}
$P \in \N$ is irreducible if and only if it is prime.
\end{tm}
\begin{proof}
Suppose $p$ is prime, i.e.
\begin{align*}
p | uv \quad \Rightarrow \quad (p|u \vee p|v)
\end{align*}
If $p|u$ then $u= kp$ and 
\begin{align*}
p= uv = (kv) p \quad \Rightarrow \quad 1 = kv \quad \Rightarrow \quad v =1
\end{align*}
Similarly if $p|v$ then $u=1$. This shows $p$ is irreducible.


Now suppose $p$ is irreducible. Suppose $p|AB$. Because $p$ is irreducible, the positive divisors of $p$ are just 1 and $p$. Therefore, either $\hcf(p,A)=1$ or $\hcf(p,A)=p$.

If $\hcf(p,A)=p$ then $p|A$ and we can close.
Suppose $hcf(p,A)=1$. 
Then $\exists x,y \in \mathbb Z$ such that:
\begin{align*}
xp + yA =1
\end{align*}
But then 
\begin{align*}
xpB + yAB =B
\end{align*} $p$ divides the first part because there is a $p$ there. $p$ divides the second part because we are assuming $p|AB$.
So $p|B$.  
\end{proof}


\begin{df}
Let $p$ be prime. Then
\begin{align*}
\forall N \in \mathbb Z, \quad \ord_p N = \max \left\{ k \in \N |\quad p^k|N \right\}
\end{align*}
= exponent of the largest power of $p$ that divides $n$.
\end{df}

\begin{ex}
\begin{align*}
\ord_2 3 & = 0 \\
\ord_2 24 & = 3
\end{align*}
\end{ex}

\begin{pr}
\begin{align*}
& \forall N_1, N_2 \in \mathbb Z &
\ord_p(N_1+N_2) & \ge \min \{ \ord_p N_1, \ord_p N2 \} \tag{1}\\
& & \ord_p(N_1,N_2) & = \ord_p N_1 + \ord_p N_2 \tag{2}
\end{align*}
\end{pr}

\begin{proof} (1)
If $p^k|N_1$ and $p^k|N_2$ then $p^k|N_1+N_2$.

(2)
\begin{align*}
a_1 = \ord_p N_i
\end{align*}
 means
\begin{align*}
p^{a_i} | N_i\quad \wedge \quad p^{a_i +1} \nmid N_i
\end{align*}
It is clear that
\begin{align*}
 (p^{a_i}|N \wedge p^{a_2}|N_2) \quad \Rightarrow \quad p^{a_1+a_2}| N_1N_2
\end{align*}
What we need to show is:
\begin{align*}
p^{a_1+a_2+1} \nmid N_1N_2
\end{align*}
Write $N_i = p^{a_i} A_i$ where $p_i \nmid A_i$. Then 
\begin{align*}
N_1N_2 = p^{a_1+a_2} A_1A_2
\end{align*}
and
\begin{align*}
p \nmid A_1, p \nmid A_2 \quad \Rightarrow \quad p \nmid A_1 A_2
\end{align*}
\end{proof}

\begin{rt}
If $p$ is a prime then $\sqrt p$ is irrational. 
\end{rt}
\begin{proof}
Suppose for contradiction that $r^2=p$ with $r=\frac k n \in \mathbb Q$.
\begin{align*}
\frac{k^2}{n^2} = p
\end{align*}
or equivalently: 
\begin{align*}
k^2 & = pn^2 \\
2 \ord_p k & = \ord_p(pn^2) = 1+ 2 \ord_p n
\end{align*}
This is a contradiction. ( A number can only be either odd or even but not both.)
\end{proof}


\begin{df}
$N \in \mathbb Z$ is a perfect square if $\exists k \in \mathbb Z: N = k^2$.
\end{df}
\begin{ec}
$N$ is a perfect square iff for all primes $p$, $\ord_p N$ is even.
\end{ec}
\begin{ex}
$\sqrt N \in \mathbb Q ~ \Leftrightarrow ~ N$ is a perfect square.
\end{ex}

\begin{proof}
$\Leftarrow$ is obvious.

We need to deal with $\Rightarrow$. Suppose $\exists r = \frac k n \in \mathbb Q$ such that $r^2 = N$.

Suppose for contradiction that $N$ is not a perfect square. i.e. there exists a prime $p$ such that $\ord_p N$ is odd. As before we write:
\begin{align*}
k^2 & = n^2 N \\
2 \ord_p k & = \ord_p k^2 = 2 \ord_p n + \ord_p N
\end{align*}
This is a contradiction because we have an even number on the left and an odd number on the right hand side.
\end{proof}


\begin{tm}
The fundamental theorem of arithmetic. \\
Every $n \in \N$ can be written uniquely in the form:
\begin{align*}
n = p_1^{a_1}  p_2^{a_2}  \dots  p_r^{a_r}
\end{align*}
where $p_1 < p_2< \dots < p_r$ are primes and $a_i \in \N \backslash {0}$ for $i\in \N \backslash\{ 0 \}$.
\end{tm}
\begin{proof}
We already know that $n$ can be written like this. For uniqueness:
\begin{align*}
a_i = \ord_{p_i} n
\end{align*}
\end{proof}

\begin{rk}
What you are really doing is:
\begin{align*}
& & n & = p_1^{a_1} \dots p_r^{a_r} \\
& \Rightarrow & \ord_{p_i} n & = \ord_{p_i} \left( p_1^{a_1} \dots p_r^{a_r} \right) \\
& & & = a_1 \ord_{p_i} p_1 + \dots a_{i-1} \ord_{p_i} p_{i-1}
+ a_i \ord_{p_i} p_i
+ a_{a+1} \ord_{p_i} p_{i+1}
+ \dots + a_r \ord_{p_i} p_r
= a_i 
\end{align*}
because $\ord_{p_i} p_i =1$ and $\ord_{p_j} p_j = 0$ if $i \neq j$.
\end{rk}

\begin{ex}
There are infinitely many primes. 
Suppose for a contradiction that:
\begin{align*}
\mathbb P = \{ \text{all primes} \} = \{ p_1< p_2< \dots < p_r \}
\end{align*}
Consider $N = p_1p_2 \dots p_r +1$. Claim $\hcf(N, p_i) =1$ for all $i= 1, \dots, r$.
Manifestly $\exists x,y \in \mathbb Z$ such that $x N + yp_i = 1$. Hence, $N$ contradicts the prime decomposition theorem.
\end{ex}

New things about hcf.
\begin{lm}
Suppose $\hcf(a,b)= 1$. Then for all $C \in \N$
\begin{align*}
(a|C \wedge b|C) & \Rightarrow ab|C \tag{1}\\
a|bC & \Rightarrow a|C \tag{2}
\end{align*}
\end{lm}
\begin{proof}
\begin{align*}
\exists x,y \in \Z:	\quad ax+by=C \tag{$*$}
\end{align*}
For (1) multiply ($*$) with $C$. We get
\begin{align*}
axC + byC = C
\end{align*}
$b|C \Rightarrow ab|aC$ so $ab$ divides $axC$. $a|C \Rightarrow ab|bC$ so $ab$ divides $byC$. Therefore, $ab | C$.

For (2) again
\begin{align*}
axC+byC = C
\end{align*}
$a$ obviously divides $axC$ and $byC$ as well because of our assumption. Thus $a|C$.
\end{proof}

\begin{tm}
Suppose $c|a$ and $c|b$. Then $c|\hcf(a,b)$.
\end{tm}

\begin{proof}
Indeed let $d = \hcf(a,b)$. We know $\exists x,y \in \Z$ such that
\begin{align*}
ax+by=d
\end{align*}
$c$ divides $ax$ as well as $bx$. Thus, $c|d$
\end{proof}
It follows easily from unique prime factorization that for all $p$, $\ord_p \hcf(a,b) = \min \{ \ord_p a, \ord_p b \}$.
Another way to say this is: if 
\begin{align*}
a  = \Pi p_i^{r_i}, b  = \Pi p_i ^{s_i} 
\end{align*}
Then
\begin{align*}
\hcf(a,b) & = \Pi p_i^{\min\{r_i,s_i\}}
\end{align*}

\begin{rk}
This is a really bad method for computing the hcf.
\end{rk}

\begin{ex} 
\begin{itemize}
\item
(a Diophantine equation) \\
Solve for $x,y \in \N$:
\begin{align*}
4y^2 = x^3+1
\end{align*}
Write this as: 
\begin{align*}
x^3 = 4y^2 -1 = (2y+1)(2y-1)
\end{align*}
Note that $\hcf(2y+1,2y-1) = c = 1$:
\begin{align*}
(c|2y+1~ \wedge ~ c|2y-1) \quad \Rightarrow \quad c| ~(2y+1)- (2y+1)= 2
\end{align*}
but both numbers are odd so $c=1$. 

Suppose $\hcf(A,B) = 1$ and $AB = x^3$ is a perfect cube. Then both $A,B$ are perfect cubes. Indeed, if $p$ prime, $\ord_p AB = 3k$ and at least one of $\ord_p A, ~\ord_p B =0$.

So $2y-1, 2y+1$ are both perfect cubes. Only small cubes can have such a small distance.
\begin{align*}
\dots \quad -27\quad -8 \quad -1 \quad 0 \quad 1 \quad 8 \quad 27 \quad \dots
\end{align*}
At the end: 
\begin{align*}
2y-1 & = -1 \\
2y+1 & = 1 \\
\Rightarrow \qquad y =0 \quad & \wedge  \quad x = -1
\end{align*}
These are all the solutions!
\item
Consider the equation
\begin{align*}
3x + 7y = 18, \qquad (x,y) \in \Z^2 \tag{1}
\end{align*}
Ordinarily, you would use Euklid's algorithm to find one solution. This solution can then help to find the general solution.
In this case it is easy to spot one: $(x,y) = (6,0)$.
Write $(x,y) = (x_0,y_0) + (x',y')$ and 
\begin{align*}
& & 3x + 7y & = 18 \\
& \Leftrightarrow & 3x' + 7y' & = 0 \tag{2}
\end{align*}
Thus, finding all solutions of (2) is the same as finding all solutions of (1).
Note: $7y' = -3x'$.
\begin{align*}
(\hcf(7,3) = 1 \quad \wedge \quad 7|3x') \qquad \Rightarrow 7|x'
\end{align*}
Therefore, there exists a $t \in \Z : x' = 7t$.
Then $7y' = -3 t$ 
The set of all solutions is precisely:
\begin{align*}
\{ (x, y)| \quad (x,y)= (6,0) + t(7,-3), \quad  t \in \Z \}
\end{align*}
We think of this as a parametrised 'integral line'.
\end{itemize}
\end{ex}

Let us regard the general case of the example above.
Suppose $A,B,C \in \Z$. We want to find all solutions $(x,y) \in \Z^2$ of:
\begin{align*}
Ax + By =C \tag{*}
\end{align*}
Sometimes we may want to look for $(x,y) \in \N^2$. 
Suppose $\hcf(a,b) = d$. Then we write 
\begin{align*}
a & = d a'\\
b & = db'
\end{align*}
with $a', b' \in \Z$ and co-prime.

\begin{rk}
$(*)$ has $a$ solutions $x_0,y_0$ iff $d = \hcf(a,b) | c$.
\end{rk}  
\begin{proof}
Suppose $ax_0 + by_0 = c$ Then $d|a$ and $d|b$ and hence $d|c$.
Vice versa suppose that $d|c$, i.e. $c =dc'$. We know that there exists $u,v \in \Z$ such that $au + bv = d$ and then $x_0 = ac'$, $y_0 = vc'$ satisfy
\begin{align*}
ax_0 + by_0 = auc' + bvc' = dc' = c
\end{align*} 
\end{proof}
\begin{rk}
Euklid's algorithm gives us a way to compute a solution $x_0, y_0$ of $(*)$ if a solution exists. The problem is to find all other solutions.
\end{rk}


Suppose that $x_0, y_0$ is a solution of $(*)$. Write 
\begin{align*}
(x,y) = (x_0, y_0) + (x',y')
\end{align*}
then $x,y$ is a solution of $(*)$ is a solution of the homogeneous equation.
\begin{align*}
ax' + bx' = 0 \tag{1}
\end{align*}

(1) is equivalent to:
\begin{align*}
a'x' + b'y' = 0 \tag{2}
\end{align*}
Now $\hcf(a', b')= 1$.
Write 
\begin{align*}
-b'y' = a'x'
\end{align*}
and notice $b'|a'x'$. Since $\hcf(b'a')= 1$ then $b'|x'$. Wo we can write $x' = tb'$ for some $t \in \Z$ Then (2) says
\begin{align*}
b'a't+b'y' = 0
\end{align*}
So 
\begin{align*}
a't + y' = 0
\end{align*}
i.e. $y' = a' t$. 
Conclusion:
The set of solutions of $(1)$ is
\begin{align*}
\{(x,y) = t(b'-a')|\quad t \in \Z \}
\end{align*}
The set of solutions of $(*)$ is 
\begin{align*}
\{(x,y) + (x_0+ y_0) = t(b'-a')| \quad t \in \Z \}
\end{align*}
This is the end of the general theory.

\begin{ex}
Find all $c \in \Z$ such that the equation
\begin{align*}
5x+11y = c
\end{align*}
has a solution $(x,y) \in \N^2$
All integer solutions $(x,y)$ are of the form $(x,y) = (x_0,y_0) + t(11,-5)$. (this idea can be obtained by drawing the graph. The slope of the line is $\frac{-11}{5}$)

If $c$ is big enough we can always find a positive solution $(x,y)$ with $x \ge 0$, $y \ge 0$. By adding/subtracting an integer multiple of the vector $(11,-5)$. This will work if:
\begin{align*}
& & \sqrt{\frac{c^2}{5^2} + \frac{c^2}{11^2}} & \ge \sqrt{11^2 + 5^2} \\
& \Leftrightarrow & \frac c {55}  \sqrt{11^2 + 5^2} & \ge \sqrt{1^2+5^2}
\end{align*}
If $c \ge 55$ then a solution $(x, y) \in \N^2$ exists.
For every pair $(x,y) \in \N$ we compute $5x + 11y$.
We can get the following possible sums smaller than 55
\begin{align*}
& 0, 5, 10, 15, 20, 25, 30, 35, 40, 45, 50, \\
& 11, 16, 21, 26, 31, 36, 41, 46, 51, \\
& 22, 27, 32, 37, 42, 47, 52, \\
& 33, 38, 43, 48, 53, \\
& 44, 49, 54
\end{align*}
Conclusion:
The equation $5x + 11y = c$ has a solution $(x,y) \in \N^2$ iff
\begin{align*}
c \in \{0,5,10,11,15,16,20,21,22,25,26,27,30,31,32,33,35,36,37, 38 \}
\quad \vee \quad c \le 40
\end{align*}
\end{ex}

Relations

\begin{df}
The \emph{Cartesian Product} of two sets $A, B$ is the set
\begin{align*}
A \times B := \{ (a,b)| \quad a \in A, b \in B \}
\end{align*}
where $(a,b)$ is an ordered pair.
\end{df}

\begin{ex}
Suppose $A = \{1, 2, 3 \}$, $B= \{ 1,2\}$. Then 
\begin{align*}
A \times B = \{(1,1), (1,2), (2,1), (2,2), (3,1), (3,2) \}
\end{align*}
\end{ex}

\begin{df}
A relation on a set $S$ is a subset $R \subseteq S \times S$. We write
\begin{align*}
aRb \quad \text{or} \quad a \sim_R b \qquad \Leftrightarrow \qquad (a,b) \in R
\end{align*}
\end{df}

\begin{ex}
\begin{itemize}
\item
Let $S$ be this room.
\begin{align*}
R = \{ ( a,b) | \quad a\text{ is in love with }b \}
\end{align*}
\item
Some mathematical examples
\begin{align*}
\triangle & = \{ (a,a) | \quad a \in S \} \\
R & = S \times S \backslash \triangle 
\end{align*}
Some relations on $\R$ are 
\begin{align*}
A &= \{(a,b)| \quad a \leq b \} \subseteq \R^2 \\
B &= \{(a,b)| \quad a < b \} \subseteq \R^2\\
C & = \{(x,y) \in \R^2| \quad y^2-x^2 = 1 \} \subseteq \R^2 
\end{align*}
\end{itemize}
\end{ex}

\begin{ex}
How many relations are there on the set $S = \{1,2\}$?
A relation on $S = \{1,2\}$ is precisely a subset of $ S \times S = \{(1,1), (1,2), (2,1), (2,2) \}$. Hence, there are $2^4 =16$ Relations on the set.
\end{ex}


\begin{rk}
If $T$ is a set with $n$ elements, then $T$ has $2^n$ subsets. (We will prove that further in these notes.)
\end{rk}


\begin{pr}
\begin{itemize}
\item
R is reflexive if for all $a \in S$, \quad $(a,a) \in \R$.
\item
R is symmetric if for all $a,b \in S$, \quad $(a,b) \in R \Rightarrow (b,a) \in R$.
\item
R is transitive if for all $a,b,c \in S$, \quad $(a,b) \in R \wedge (b,c) \in R \Rightarrow (a,b) \in R$ .
\item
R is an equivalence relation if R is reflexive, symmetric and transitive 
\end{itemize}
\end{pr}

Now we have the language to speak about modular arithmetic:


Fix $m \in \N$. Then we define a relation $R$ on $\Z$:
\begin{align*}
R = \{ (a,b) \in \Z \times \Z| \quad m|(a-b) \}
\end{align*}
Notation
\begin{align*}
a \sim_R b \qquad \Leftrightarrow \qquad a \equiv b \quad \mod m
\end{align*}
read $a$ congruent to $B$ mod $n$. This is the same as saying:

Dividing $m$ into $a$ or $b$ leaves the same remainder.

\begin{pr}
\begin{align*}
& \forall a \in \Z & a & \equiv a \mod m \\
& \forall a,b \in \Z & m |a-b \quad & \Leftrightarrow \quad m |b-a \\
& \forall a,b,c \in \Z & m|a-b \wedge m|b-c \quad & \Rightarrow  \quad m |(a-b) + (b-c) = (a-c)
\end{align*}
Therefore, $\equiv \mod m$ is an equivalence relation.
\end{pr}

\begin{df}
Let $\sim$ be an equivalence relation on a set $S$. For all $a \in S$ we define 
\begin{align*}
[a] = \{ b \in S| \quad a \sim b \}
\end{align*}
\end{df}

\begin{tm}
\begin{align*}
& a \sim b  & & \Leftrightarrow & \quad [a] & =[b] \tag{1}\\
& a \nsim b & & \Leftrightarrow & \quad [a] \cap [b] & = \emptyset \tag{2}
\end{align*}
\end{tm}
\begin{proof}
(1). first prove $ \Rightarrow$. We show first $[a] \subseteq [b]$.

Suppose $a \sim c$ then $c \sim a$ and $a \sim b \Rightarrow c \sim b \Rightarrow b \sim c$. So $c \in [b]$.

Similarly $[b] \subseteq[a]$. 

$\Leftarrow$ is pretty obvious.

(2). Suppose $[a]\cap [b] \neq \emptyset$. so let $c \in [a] \cap [b]$. Hence, $a \sim c, b \sim c$. Therefore $a \sim c $ and $c \sim b$ and thus $ a \sim b$
So $[a] = [b]$. Please finish the proof by yourself.
\end{proof}

\begin{df}
The \emph{quotientset} of $S$ by an equivalence relation $R$ is
\begin{align*}
S/R := \left\{ [a]~| \quad  a \in S \right\} \subseteq P(S)
\end{align*}
where $P(S) = \{A | \quad A \subseteq S \}$.
\end{df}

Suppose that we know how to choose a unique distinguished element in each equivalence class on a set $S$:
denote by $\overline a$ the distinguished element in $[a]$. (So if $b \in [a]$ then $\overline a = \overline b$. Then we can form a concrete model
\begin{align*}
S/R = \{ \overline a ~ | \quad a \in S \} \subseteq S
\end{align*}

\subsection{Modular Arithmetic}
Fix $n \geq 2, m \in \N$
\begin{align*}
a \equiv b \mod n
\end{align*}
iff $m|a-b$.
The quotientset is denoted $\Z/m\Z$.
\begin{align*}
\forall a \quad \exists ! r \quad 0 \leq r < m: \qquad a = qm +r
\end{align*}
We call this $r$ the smallest residue of $a \mod m$, we denote it by $\overline a$.
So 
\begin{align*}
\overline a = \overline b \quad \Leftrightarrow \quad a \equiv b \mod m
\end{align*}
\begin{ex}
	\begin{align*}
	\overline 7 = 1 = \overline{10} \mod 3
	\end{align*}
\end{ex}
Every equivalence class has a unique distinguished representative in $\{0,1,2,3, \dots, n-1\} \subseteq \Z$. so we can think of $\Z/m\Z$ as 'being' the set $\{0,1,2,3, \dots, n-1\}$.

\begin{tm}
	Suppose $m \in \N$
	\begin{align*}
	a & \equiv b \mod m \\
	c & \equiv d \mod m \\
	\Rightarrow \qquad a+c & \equiv b+d \mod m \\
	\wedge \qquad a \cdot c & \equiv b \cdot d \mod m
	\end{align*}
	This defines the operations $+, \cdot$ on $\Z/m\Z$.
\end{tm}

\begin{pr}
These operations have familiar rules:
\begin{align*}
a+b  & \equiv b+a & \mod m \\
a \cdot b & \equiv b \cdot a & \mod m \\
a \cdot (b+c) & \equiv a \cdot b + a \cdot c & \mod m
\end{align*}
\end{pr}

\begin{ex}
	Addition and multiplication tables in $\Z/m\Z$. Addition table: 
	\begin{center}
	\begin{tabularx}{.4\textwidth}{>{\bfseries}+Z^Z^Z^Z^Z^Z^Z}
		\toprule
		\rowstyle{\bfseries} & 0 & 1 & 2 & 3 & 4 & 5 \\ \toprule
		0                    & 0 & 1 & 2 & 3 & 4 & 5 \\
		1                    & 1 & 2 & 3 & 4 & 5 & 0 \\
		2                    & 2 & 3 & 4 & 5 & 0 & 1 \\
		3                    & 3 & 4 & 5 & 0 & 1 & 2 \\
		4                    & 4 & 5 & 0 & 1 & 2 & 3 \\
		5                    & 5 & 0 & 1 & 2 & 3 & 4 \\ \bottomrule
	\end{tabularx}
\end{center}
	Multiplication table: 
	\begin{center}
		\begin{tabularx}{.4\textwidth}{>{\bfseries}+Z^Z^Z^Z^Z^Z^Z}
			\toprule
			\rowstyle{\bfseries} & 0 & 1 & 2 & 3 & 4 & 5 \\ \toprule
			0                    & 0 & 0 & 0 & 0 & 0 & 0 \\
			1                    & 0 & 1 & 2 & 3 & 4 & 5 \\
			2                    & 0 & 2 & 4 & 0 & 2 & 4 \\
			3                    & 0 & 3 & 0 & 3 & 0 & 3 \\
			4                    & 0 & 4 & 2 & 0 & 4 & 2 \\
			5                    & 0 & 5 & 4 & 3 & 2 & 1 \\ \bottomrule
		\end{tabularx}
	\end{center}
\end{ex}

The thing to note here is: In $\Z/6\Z$ $\overline 2, \overline 3 \neq 0$ but $\overline 2 \cdot \overline 3 = 0$.

\begin{ex}
		Multiplication table in $\Z/5\Z$: 
	\begin{center}
		\begin{tabularx}{.33\textwidth}{>{\bfseries}+Z^Z^Z^Z^Z^Z}
			\toprule
			\rowstyle{\bfseries} & 0 & 1 & 2 & 3 & 4 \\ \toprule
			0                    & 0 & 0 & 0 & 0 & 0 \\
			1                    & 0 & 1 & 2 & 3 & 4 \\
			2                    & 0 & 2 & 4 & 1 & 3 \\
			3                    & 0 & 3 & 1 & 4 & 2 \\
			4                    & 0 & 4 & 3 & 2 & 1 \\ \bottomrule
		\end{tabularx}
	\end{center}
\end{ex}

\begin{rk}
	In $\Z/5\Z$
	\begin{align*}
	(a \neq 0 \quad \wedge \quad b \neq 0) \qquad \Leftrightarrow \qquad ab \neq 0
	\end{align*}
	In $\Z/5$ every $a \neq 0$ has a multiplicative inverse $a^{-1}$:
	\begin{align*}
	1^{-1} = 1 \quad 2^{-1} = 3, \quad 3^{-1} = 2, \quad 4^{-1} = 4
	\end{align*} 
\end{rk}

\begin{tm}
		In $\Z/m\Z$ every nonzero element has a multiplicative inverse iff $m$ is prime.
\end{tm}

\begin{proof}
	Suppose $m$ is prime. Let $a \neq 0$ in $\Z/p\Z$. To find a multiplicative inverse of a is exactly the same as solving 
	\begin{align*}
	ax = 1 \qquad \text{in} ~\Z/p\Z
	\end{align*}
	that is $ax-1 = py$. That is: Find $x,y \in \Z$ such that $ax - py =1$.
	We can do that iff $\hcf(a,p)=1$. Since $p$ is prime this is the same as $p \nmid a$ and this is the same as saying $a \not\equiv 0 \mod p$.
	
	On the other hand if $m$ is not prime, then there exist $m_1, m_2$ with $m = m_1 \cdot m_2$ and $1 < m_1, m_2 <m$.
	\begin{align*}
	m_1 & \not\equiv 0 & \mod m \\
	m_2 & \not\equiv 0 & \mod m 
	\end{align*}
	but
	\begin{align*}
	m_1 \cdot m_2 \equiv 0 & \mod m
	\end{align*}
	So neither $m_1$ nor $m_2$ have a multiplicative inverse.
	
	Indeed if there was a $u_1$ such that
	\begin{align*}
	u_1 m_1 \equiv 1 & \mod m
	\end{align*}
	then 
	\begin{align*}
	(u_1 m_1) m_2 \equiv m_2 & \not\equiv 0 & \mod m \\
	= u_1 (m_1m_2) & \equiv 0 & \mod m
	\end{align*}
\end{proof}

\begin{ex}
	\begin{itemize}
		\item
		What is $2^{100} \mod 15$?
		\begin{align*}
		2^4 \equiv 1 \mod 15 
		\end{align*}
		so $2^{100} = (2^4)^{25} \equiv 1 \mod 15$.
		\item
		Compute $5^{67} \mod 14$. We first have to find the binary expansion of 67 which is 1000011. Idea:
		\begin{align*}
		67 & = 64 +2+1 & \\
		& = 2^6 + 4+1 \\
		5^2 & \equiv 11 & \mod 14\\
		5^4 & \equiv (5^2)^2 = 9 & \mod 14\\
		5^8 & \equiv 9^2 = 11 & \mod 14 \\
		5^{16} & \equiv 11^2 = 9 & \mod 14 \\
		5^{32} & \equiv 11 & \mod 14 \\
		5^{64} & \equiv 9 & \mod 14 \\
		5^{67} = 5^{64} \cdot 5^2 \cdot 5 & \equiv 9 \cdot 11 \cdot 5 \equiv 5 & \mod 14
		\end{align*}
		
		\item
		Prove that
		\begin{align*}
		\forall n \in \N, \qquad \sqrt{5n+3} \not\in \Q
		\end{align*}
		All we need to show is that for all $n \in \N, 5n+3$ is not a perfect square, i.e. for all $n \in \N$ the equation 
		\begin{align*}
		5n+3 = x^2
		\end{align*}
		has no interger solution $x \in \Z$. Something stronger is true:
		\begin{align*}
		x^2 & \equiv 3 & \mod 5
		\end{align*}
		has no solution in $\Z/5\Z$.
		\begin{align*}
		x \equiv 0 \quad \mod 5 \qquad & \Rightarrow \qquad x^2 \equiv 0 \quad \mod 5 \\
		x \equiv 1 \quad \mod 5 \qquad & \Rightarrow \qquad x^2 \equiv 1 \quad \mod 5 \\
		x \equiv 2 \quad \mod 5 \qquad & \Rightarrow \qquad x^2 \equiv 4 \quad \mod 5 \\
		x \equiv 3 \quad \mod 5 \qquad & \Rightarrow \qquad x^2 \equiv 4 \quad \mod 5 \\
		x \equiv 4 \quad \mod 5 \qquad & \Rightarrow \qquad x^2 \equiv 1 \quad \mod 5 \\
		\end{align*}
		So no $x$ satisfies $x^2 \equiv 3 \mod 5 $.
		\begin{rk}
			The same argument also shows that for all $n \in N, 5n+2$ is not a perfect square
		\end{rk}
		\item
		Show that the only solution of
		\begin{align*}
		x^2 + 5y^2 = 3z^2
		\end{align*}
		for $x,y,z \in \Q$ is the trivial solution $x=0, ~ y=0~, z=0$.
		
		The first thing to note is: If $x_0, y_0, z_0$ is a solution and $r\in \Q$, then $r x_0, ry_0, rz_0$ is also a solution.
		Suppose for contradiction that $x_0, y_0, z_0 \in \Q$ is a nontrivial solution. 
		Multiplying $x_0, y_0, z_0$ by a common denominator, we may assume that $x_0, y_0, z_0 \in \Z$ and $\hcf(x_0, y_0, z_0) = 1$. In particular then reducing the equation $\mod 5$ we get $x^2 \equiv 3z^2 \mod 5$. If $z \equiv 0 \mod 5$ then also $x \equiv 0 \mod 5$.
		Assume $z_0 \not\equiv 0 \mod 5$. Then $z$ has a multiplicative  inverse $\mod 5$. I.e. there exists a $w_0 \in \Z$ such that 
		\begin{align*}
		z_0 w_0 \equiv 1 \mod 5
		\end{align*}
		Multiply by $w_0^2$:
		\begin{align*}
		(x_0w_0)^2 \equiv 3(z_0 w_0)^2 \equiv 3 \mod 5
		\end{align*}
		and this is impossible.
		So $x_0, z_0$ are both $\equiv 0 \mod 5 $
		\begin{align*}
		x_0 = 5 x_0', z_0 = 5 z_0'
		\end{align*}
		plug back into (*)
		\begin{align*}
		25 x_0'^2 + 5y_0^2 = 75 x_0'x_0'^2
		\end{align*} 
		Divide then by 5:
		\begin{align*}
		5 x_0'^2 + y_0^2 = 15 z_0'^2 \\
		y_0^2 = 5 (3z_0'^2 - x_0'^2)
		\end{align*}
		So $5|y_0^2$ and  $5 |y_0$ as well. In fact $5|x_0,y_0z_0$. This is a contradiction because:
		\begin{align*}
		\hcf(x_0,y_0z_0) = 1
		\end{align*}
		Hence every solution is trivial.
	\end{itemize}
\end{ex}

\begin{pp}
The equation
\begin{align*}
ax \equiv 1 \mod m
\end{align*}
is solvable for $x$ if and only if $\hcf(a,m) = 1$. I.e. $a$ has a multiplicative inverse $\mod m$ if and only if $\hcf(a,m)=1$
\end{pp}

\begin{proof}
	\begin{align*}
	ax \equiv 1 \mod \qquad \Leftrightarrow \qquad \exists y: \quad ax+ my = 1
	\end{align*}
	The equation is solvable if and only if
	\begin{align*}
	\exists x,y \in \Z: \quad ac + my = 1
	\end{align*}
	We know this is equivalent to $\hcf(a,m)=1$.
\end{proof}

\begin{rk}
$\hcf(a,m)$ only depends on $[a] \mod m$. i.e. for all $k \in \Z$:
\begin{align*}
\hcf(a+kn, m ) = \hcf(a,m)
\end{align*}
\end{rk}

\begin{df}
Let $A$ be a set. Denote the number of elements of $A$, $|A| = \# A$. 
\end{df}

\begin{ex}
	$\displaystyle |\Z/m\Z| = m$.
\end{ex}

\begin{df}
Let $(\Z/m\Z)^x$ be the set of $[a]$ which have a multiplicative inverse. That means:
\begin{align*}
(\Z/m\Z)^x =\{r|~0 \le r < m \wedge \hcf (r,m) = 1 \}
\end{align*}
Furthermore, define Euler's function $\varphi$ with
\begin{align*}
\varphi (m) := \left| ( \Z/m \Z)^x \right| .
\end{align*}
\end{df}

\begin{ex}
	 If $p$ is prime then for all $a \in \Z$ either $p|a$ or $\hcf(p,a)=1$ so $\varphi(p) = p-1$. We will soon know how to compute $\varphi(m)$ for all $m \in \N$.
\end{ex}

\begin{ec}
If $a,b,c \in \Z$, define $\hcf(a,b,c) = \max \{t \in \Z | t|a \wedge t|b \wedge t|c\}$. Prove 
\begin{itemize}
	\item
	$\hcf(a,b,c) = \hcf(\hcf(a,b),c)$.
	\item
 there exist integers $p,q,s \in \Z$ such that 
\begin{align*}
\hcf(a,b,c) = ap + bq + cs
\end{align*}
\item
Use unique factorization to show:
\begin{align*}
\forall p ~\text{prime} \qquad \ord_p \hcf (a,b,c) = \min \left\{\ord_p a, \ord_p b, \ord_p c \right\}
\end{align*}
\end{itemize}
\end{ec}














