

% Office hours mondays 17:00 Huxley 673
\section{Sets}
\begin{df}
A \emph{set} $S$ is a collection of objects (called \emph{elements} of the set). If $x$ is an \emph{element} of $S$ let us write $x \in S$ otherwise $x \notin S$.
\end{df}
\begin{rk}
The order of the elements or any repetition is unimportant.
\end{rk}
\begin{ex} \qquad 
$\displaystyle \{1,3 \} = \{ 3,1,1 \}$
\end{ex}
\begin{df}
For two sets $S$ and $T$ let us write $S \subseteq T$ ($S$ is a \emph{subset} of in $T$) if 
\begin{align*}
x \in S \Rightarrow x \in T.
\end{align*}
\end{df}

\begin{df}
$S=T$ iff $S \subseteq T$ and $T \subseteq S$.
\end{df}

\begin{ax} \emph{Foundation Axiom} 
\begin{align*}
S \notin S
\end{align*}
\end{ax}

\begin{rk}
Nonetheless, elements can be sets.
\end{rk}

\begin{df}
$\emptyset$ is the set with no elements.
\end{df}

\begin{pr}
$\emptyset \subseteq S$ and $S \subseteq S$ for all sets S
\end{pr}

\subsection{Set Operators}

\def\firstcircle{(0,0) circle (1.5cm)}
\def\secondcircle{(0:2cm) circle (1.5cm)}

\colorlet{circle edge}{black!60}
\colorlet{circle area}{black!30}

\tikzset{filled/.style={fill=circle area, draw=circle edge, thick},
	outline/.style={draw=circle edge, thick}}


\begin{df}
The \emph{intersection} $S \cap T$ of two sets $S$ and $T$ is 
\begin{align*}
\{ x | ~ x \in S ~\text{and}~ x \in T \}.
\end{align*} 
\end{df}
\begin{df}
The \emph{union} $S \cup T$ of two sets $S$ and $T$ is 
\begin{align*}
\{ x | ~ x \in S ~\text{or}~ x \in T \}.
\end{align*} 
\end{df}
\begin{df}
The \emph{difference} $S \backslash T$ of two sets $S$ and $T$ is 
\begin{align*}
\{ x | ~ x \in S ~\text{and}~ x \not\in T \}.
\end{align*} 
\end{df}
\begin{df}
	The \emph{symmetric difference} $S \triangle T$ of two sets $S$ and $T$ is 
	\begin{align*}
	\{ x | ~ x \in S ~\text{or}~ x \in T ~\text{but not both} \}.
	\end{align*} 
\end{df}
	\begin{df}
		The complement $A^C$ of a set $A$ with regard to a set $\Omega, A \subseteq \Omega$ is
		\begin{align*}
		\{ x\in \Omega| ~x \notin A \}
		= \Omega \backslash A.
		\end{align*}
	\end{df}
	\begin{rk}
		The complement is only used when the reference set $\Omega$ is clear.
	\end{rk}
\begin{figure}[h]
	\caption{The operations on sets can be visualized by \emph{Venn diagrams}. In order to understand a set-related problem, in some cases it may be helpful to draw a Venn Diagram.}
	%A intersection B
	\centering
	\begin{tikzpicture}
	\begin{scope}
	\clip \firstcircle;
	\fill[filled] \secondcircle;
	\end{scope}
	\draw[outline] \firstcircle node {$A$};
	\draw[outline] \secondcircle node {$B$};
	\node[anchor=south] at (current bounding box.north) {$A \cap B$};
	\end{tikzpicture}
	\quad
	%A union B
	\begin{tikzpicture}
	\draw[filled] \firstcircle node {$A$}
	\secondcircle node {$B$};
	\node[anchor=south] at (current bounding box.north) {$A \cup B$};
	\end{tikzpicture}
	\quad
	%A without B
	\begin{tikzpicture}
	\begin{scope}
	\clip \firstcircle;
	\draw[filled, even odd rule] \firstcircle node {$A$}
	\secondcircle;
	\end{scope}
	\draw[outline] \firstcircle
	\secondcircle node {$B$};
	\node[anchor=south] at (current bounding box.north) {$A \backslash B$};
	\end{tikzpicture}
	\\
	%B without A
	\begin{tikzpicture}
	\begin{scope}
	\clip \secondcircle;
	\draw[filled, even odd rule] \secondcircle node {$B$}
	\firstcircle;
	\end{scope}
	\draw[outline] \secondcircle
	\firstcircle node {$A$};
	\node[anchor=south] at (current bounding box.north) {$B \backslash A$};
	\end{tikzpicture}
	\quad
	%A symmetric difference B
	\begin{tikzpicture}
	\draw[filled, even odd rule] \firstcircle node {$A$}
	\secondcircle node{$B$};
	\node[anchor=south] at (current bounding box.north) {$A \triangle B$};
	\end{tikzpicture}
	\quad
	%A complement
	\begin{tikzpicture}[even odd rule,rounded corners=2pt]
	\filldraw[filled] (0,0) rectangle (5,3.2) node[anchor=south] at (current bounding box.north) {$A^C$}
	(1.6,1.6) circle (1.5cm) node {$A$};
	\draw (4, 1.6) node {$\Omega$};
	\end{tikzpicture}
\end{figure}

Some sets we will work with in this course are:
\begin{align*}
\mathbb N & = \{0, 1, 2, \dots \} \\
\mathbb Z & = \{0, 1, -1, 2, -2 \dots \} \\
\mathbb Q & = \left. \left\{ \frac p q \right| ~ p \in \mathbb N, q \in \mathbb Z \backslash \{0\} \right\} \\
\R & ~\text{reals} \\
\mathbb C & ~\text{complex numbers}
\end{align*}


\subsection{Intervals in $\R$}
\begin{df}
If $a,b \in \R, ~a \le b$, then
\begin{align*}
[a,b] & = \{ t \in \R| ~a \le t \le b \} \\
(a,b) & = \{ t \in \R| ~a < t < b \} \\
[a,b) & = \{ t \in \R| ~a \le t < b \} \\
(a,b] & = \{ t \in \R| ~a < t \le b \} \\
[a, \infty) & = \{ t \in \R| ~a < t \} \\
( -\infty ,b] & = \{ t \in \R| ~t \le b \}.
\end{align*}
\end{df}


\subsection{Infinite Unions and Intersections}
\begin{df}
Suppose that, for all $n \in \mathbb N$, we are given a set $A_n$. Define
\begin{align*}
\bigcup_{n=a}^\infty A_n & = \left\{ x | ~\text{there exists a }n\in \mathbb N, n\ge a ~\text{ such that } x \in A_n \right\} \\
\bigcap_{n=a}^\infty A_n & = \{ x | ~\text{for all }n\in \mathbb N, n\ge a ~ \text{such that } x \in A_n \} .
\end{align*}
\end{df}

\begin{ex}
\begin{align*}
\bigcup_{n=1}^\infty \left[0,1-\frac 1 n \right] & = [0,1 ) \\ 
\bigcap_{n=1}^\infty \left(1- \frac 1 n, 1 + \frac 1 n \right) & = \{1\}
\end{align*}
\end{ex}





 
 
 
 
 
 