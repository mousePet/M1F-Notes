\section{Proofs}
\subsection{Elements of propositional calculus}
\begin{df}
A \emph{statement} (proposition) is an assertion that can be either true (T) or false (F).
\end{df}

\begin{rk}
In maths such an assertion usually takes the form:
"If such and such assumptions are made, then we can infer such and such conclusions."
 \end{rk}

\begin{ex} 
\begin{itemize}
\item
$n  = 3$
\item
$(A+B)^2  = A^2 + 2AB + B^2$
\item
If it $n^2$ is odd, then $n$ is odd too.
\item
If it rains, then it is cloudy.
\item
For all real numbers greater than or equal to 0 there exists a square root. 
\end{itemize}
\end{ex}



\begin{df}
A \emph{proof} is a chain of statements linked by logical implications (inferences) that establish the truth of the last statement. In the course of the proof one is allowed to "call up" 
\begin{itemize}
\item
 assumptions that are made,
\item
 statements proven previously,
\item
 axioms (statements that are generally accepted and never proven).
\end{itemize}
\end{df}



"Grammar elements" of mathematical statements are Quantifiers:
\begin{center}
\begin{tabularx}{.65\textwidth}{XXX}
\toprule
Type & Sign & Meaning \\
\toprule
Existential & $\exists$ & there exists \\
& $\exists_1$ & there exists a unique \\
\midrule
Universal & $\forall$ & for all \\
\midrule
& $:$, $|$ & such that \\
\bottomrule
\end{tabularx}
\end{center}
Ways to form new statements from old ones are:
\begin{itemize}
\item
If $P$ is a statement then $\overline P$ "non-P" is the statement which is true if $P$ is false and false if $P$ is true.
\item
If $P$ and $Q$ are statements then we can form:
\begin{center}
\begin{tabularx}{.6\textwidth}{XX}
\toprule
Sign & Meaning \\
\toprule
$P \wedge Q$, $P\& Q$ & $P$ and $Q$. \\
$P \vee Q$ & Either $P$ or $Q$ or both. \\
$P ~\underline{\vee} ~Q$ & Either $P$ or $Q$ but not both. \\ 
\midrule
$P \Rightarrow Q$ & If $P$ then $Q$. \\
$P \Leftarrow Q$ & If $Q$ then $P$. \\  
$P \Leftrightarrow Q$ & $P$ if and only if $Q$. \\
\bottomrule
\end{tabularx}
\end{center}

\end{itemize}

\begin{rk}
$P \Rightarrow Q$ means any of the following:
\begin{itemize}
\item
If $P$ then $Q$.
\item
$Q$ if $P$.
\item
$P$ is true only if $Q$ is true.
\item
$P$ only if $Q$.
\item
$P$ is sufficient for $Q$.
\item
$Q$ is necessary for $P$.
\item
If $Q$ is false then $P$ is false.
\item
$\overline{Q} \Rightarrow \overline{P}$
\end{itemize}
Similarly, $P \Leftrightarrow Q$ means any of the following:
\begin{itemize}
\item
$(P \Rightarrow Q) \wedge (Q \Rightarrow P)$
\item
$P$ if and only if $Q$.
\item
$P$ is necessary and sufficient for $Q$.
\end{itemize}
\end{rk}

The rigorous definition of $P \wedge Q$, $P \Rightarrow Q$ can be made through a truth table.

\begin{df}
$P \wedge Q$ is defined by:
\begin{center}

\begin{tabularx}{.25\textwidth}{XXX}
\toprule
$P$ & $Q$ & $P \wedge Q$ \\
\toprule
T & T & T \\
T & F & F \\
F & T & F \\
F & F & F \\
\bottomrule
\end{tabularx}
\end{center}
\end{df}
\begin{df}
Also, $P \Rightarrow Q$ is defined by:
\begin{center}
\begin{tabularx}{.27\textwidth}{XXX}
\toprule
$P$ & $Q$ & $P \Rightarrow Q$ \\
\toprule
T & T & T \\
T & F & F \\
F & T & T \\
F & F & T \\
\bottomrule
\end{tabularx}
\end{center}
\end{df}

\begin{ex}
The statement 
"If $x \in \{n \in \mathbb N | ~n^2 <0 \}$, then $x$ is a sheep."
is true as well as the statement
"If $x \in \{n \in \mathbb N | n^2 <0 \}$, then $x$ is not a sheep."
\end{ex}
\subsection{Inference rules}
\begin{ex}
\emph{Premise 1.} If it is raining then it is cloudy. \\
\emph{Premise 2.} It is raining. \\
\emph{Conclusion.} It is cloudy. \\

We can write this more abstractly as follows: \\
$P$: it is raining \\
$Q$: it is cloudy 

In this form: \\
\emph{Premise 1.} $P \Rightarrow Q$ \\
\emph{Premise 2.} $P$ \\
\emph{Conclusion.} $Q$
\end{ex}

This is an example of an inference rule which we write like this:
\begin{align*}
((P \Rightarrow Q) \quad \wedge \quad P) \qquad \Rightarrow \qquad Q
\end{align*}

There are other inference rules:
\begin{align*}
\left(\left(P \Rightarrow Q\right) \quad \wedge \quad \left(Q \Rightarrow R\right)\right) \qquad & \Rightarrow \qquad \left(P \Rightarrow R\right) \\
\left(\left(P \vee Q\right) \quad \wedge \quad \overline P \right) \qquad & \Rightarrow \qquad Q \\
\left(P \wedge Q\right)  \qquad &  \Rightarrow \qquad P \\
\left(\left(P \Rightarrow Q\right) \quad \vee \quad \left(P \Rightarrow R\right)\right)  \qquad &  \Rightarrow \qquad (P \Rightarrow \left(Q \vee R\right)) \\
\left(\left(P \vee Q\right) \quad \wedge \quad \left(P \Rightarrow \left(P \wedge Q\right)\right)\right)  \qquad &  \Rightarrow \qquad \left(R  \Rightarrow R\right) \\
\left(\left(P \Rightarrow Q\right) \quad \wedge \quad \left(P \Rightarrow \overline Q\right)\right) \qquad &  \Rightarrow  \qquad \overline{P} \\
(P \quad \wedge \quad (Q \vee R)) \qquad &  \Rightarrow \qquad  (P \wedge Q) \quad \vee \quad (P \wedge R)
\end{align*}

\begin{ec}
 Proof that 
\begin{align*}
\forall n \in \mathbb N \qquad n^2 \text{odd} \quad \Rightarrow \quad n~ \text{odd}  .
\end{align*}
\end{ec}

\begin{ex}
Is the following a valid argument: 
\begin{enumerate}
\item
If a movie is not worth seeing, then it is not made in the UK.
\item
A movie is worth seeing only if Prof Corti reviews it.
\item
"The Maths Graves" was not not reviewed by Prof Corti.
\item
Therefore, "The Maths Graves" is not made in the UK. 
\end{enumerate}

In order to determine this, let us rewrite the argument in a more formal way:
\begin{center}
\begin{tabularx}{.6\textwidth}{XX}
\toprule
Variable & Meaning \\
\toprule
$M$ & the set of all movies \\
$W(x)$ & "$x$ is worth seeing." \\
$UK(x)$ & "$x$ is made in the UK." \\
$C(x)$ & "Professor Corti reviews $x$." \\
$m$ & "The Maths Games" $\in M$ \\
\bottomrule
\end{tabularx}
\end{center}
Now the argument can be expressed as:
\begin{align*}
 & \forall x \in M: & \overline{W(x)} & \Rightarrow \overline{UK(x)} \tag 1 \\
 & \forall x \in M: & W(x) & \Rightarrow C(x) \tag 2 \\
 & & \overline{C(m)} \tag 3\\
 & ((1) \wedge (2) \wedge (3)) \Rightarrow & \overline{UK(x)}
\end{align*}
Yes it is a valid argument. Indeed, it is the same as:
\begin{align*}
& \forall x \in M: & UK(x) & \Rightarrow W(x) \\
& \forall x \in M: & W(x) & \Rightarrow C(x)
\end{align*}
Then you say:
\begin{align*}
& \forall x & \in M \overline{C(x)} & \Rightarrow \overline{UK(x)} \\
& & \overline{C(m)} \\
& & & \Rightarrow \overline{UK(m)}
\end{align*}
\end{ex}


\begin{rt}
What can we learn from this?
If we want to be understood, we have to learn to present our arguments better.
For instance, try to put everything in the positive. Use "if then" throughout.
A better way of writing would be:
\begin{enumerate}
\item
If $x$ is made in the UK, then $x$ is worth seeing.
\item
If $x$ is worth seeing then Prof Corti reviews it.
\item
Prof Corti did not review $m$.
\item
Therefore $m$ is not made in the UK.
\end{enumerate}
\end{rt}

\subsection{Proof-Practice}
\begin{tm}
Let $A$, $B$, $C$, $\Omega$ be sets with $A,B \in \Omega$. Then:
\begin{align*}
A \cap (B \cup C) & = (A \cap B) \cup (A\cap C) \tag{1}\\
A \cup (B \cap C) & = (A \cup B) \cap (A\cup C) \tag 2\\
(A \cup B)^C & = A^C \cap B^C \tag 3 \\
(A \cap B)^C & = A^C \cup B^C \tag 4
\end{align*}
\end{tm}
\begin{ec}
Draw Venn diagrams of these statements.
\end{ec}
\begin{proof}
Consider (1). We show first:
\begin{align*}
A \cap ( B \cup C) \subseteq (A \cap B) \cup (A \cap C)
\end{align*}
Suppose that $x \in A \cap (B \cup C)$, then $x \in A$ and $x \in B\cup C$.
That means:
\begin{align*}
& & x & \in A  \quad \wedge \quad ( x \in B \vee x \in C) \\
& \Leftrightarrow & x & \in A \cap B \quad \vee \quad x \in A \cap C \\
& \Leftrightarrow & x  & \in (A \cap B) \cup (A \cap C)
\end{align*} 
This shows $\subseteq$.
Now we show:
\begin{align*}
A \cap (B \cup C) \supseteq (A\cap B) \cup (A \cap C)
\end{align*}
Suppose $x \in (A\cap B) \cup (A \cap C)$, then $x \in A \cap B$ or $x \in A \cap C$. We now distinguish between two cases:
\begin{enumerate}
\item
$x \in A \cap B$. Then $x \in A$ and $x \in B$. Therefore, $x \in A$ and $x \in B \cup C$. Hence, $x \in A \cap (B \cup C)$.
\item
$x \in A \cap C$. Then $x \in A$ and $x \in C$. Therefore, $ x \in A$ and $x \in B \cup C$. Hence, $ x \in A \cap (B \cup C)$.
\end{enumerate}

\begin{rk}
We split the proof of $C$ in two cases. In doing so we used the inference rule:
\begin{align*}
((P \vee Q) \wedge (P \Rightarrow R) \wedge (Q \Rightarrow R)) \Rightarrow R
\end{align*}
\end{rk} 
Please finish the proof of the other statements in your own.
\end{proof}


\begin{ax} \emph{Archimedeon Axiom}
\begin{align*}
\forall r \in \R~ \exists n \in \N : \quad n>r
\end{align*}
\end{ax}

\begin{lm}
\begin{align*}
\forall a,x \in \R  \qquad \left(\forall n \in \N, n\geq 1 \quad x \geq a - \frac 1 n \right) \quad \Rightarrow \quad x \geq a
\end{align*}
\end{lm}
\begin{proof}
We argue by contradiction. Hence, we want to show
\begin{align*}
\left(\exists n \in \N, n\geq 1: \quad x < a - \frac 1 n \right) \quad \Leftarrow \quad x \leq a.
\end{align*}
By the Archimedeon Axiom
\begin{align*}
\exists n: \quad n> \frac 1 {a-x}.
\end{align*}
And then also
\begin{align*}
\frac 1 n < a-x .
\end{align*}
Therefore
\begin{align*}
x < a - \frac 1 n.
\end{align*}
\end{proof}

\begin{pp}
\begin{align*}
\bigcup_{n=1}^\infty \left[0,1-\frac 1 n \right] & = [0,1) \tag 1\\ 
\bigcap_{n=1}^\infty \left(1- \frac 1 n, 1 + \frac 1 n \right) & = \{1\} 
\tag 2
\end{align*}
\end{pp}

\begin{proof}
(2)  \\
By definition 
\begin{align*}
\bigcap_{n=1}^\infty A_n = \{ a| ~\forall n,  a \in A_n \}.
\end{align*}
Then
\begin{align*}
L= \bigcap_{n=1}^\infty  \left( 1- \frac 1 n, 1+ \frac 1 n\right) = \left\{ x \in \R \left|~ \forall n \in \N, ~ 1 -\frac 1 n < x < 1 + \frac 1 n \right\} \right. .
\end{align*}
Clearly $1 \in L$. This proofs $\supseteq$.

No we are going to prove $\subseteq$. We need to show
\begin{align*}
\left(\forall n \in \N, \quad 1- \frac 1 n < x < 1 +\frac 1 n \right) \quad \Rightarrow \quad x=1.
\end{align*}
By the lemma, $x \geq 1$. There is a similar lemma that states
\begin{align*}
\left( \forall n, \quad x \leq 1 + \frac 1 n \right) \quad \Rightarrow \quad x \leq 1 .
\end{align*}
So in fact $x \geq 1$ and $x \leq 1$. Thus $x=1$. This shows $\subseteq$.\\
(1) \\
Recall that by definition
\begin{align*}
\bigcup_{n=1}^\infty A_n = \{a| ~ \exists n : ~ a \in A_n \} .
\end{align*}
It is easy to see
\begin{align*}
\bigcup_{n=1}^\infty \left[ 0, 1 - \frac 1 n \right) \subseteq [0,1) 
\end{align*}
Indeed if
\begin{align*}
\exists n : \quad 0 \leq x \leq 1 -\frac 1 n,
\end{align*}
then
\begin{align*}
0 \leq x < 1.
\end{align*}
This shows $\supseteq$. Next we show $\subseteq$. This means exactly
\begin{align*}
(0 \leq x < 1) \Rightarrow \left( \exists n: ~ 0 \leq x \leq 1 - \frac 1 n \right).
\end{align*}
By the Archimedeon axiom 
\begin{align*}
\exists n: \quad n > \frac 1 {1-x}.
\end{align*}
Hence $\frac 1 n < 1-x$ and then $x< 1-\frac 1 n$.
\end{proof}



\begin{tm}
	\begin{align*}
	\forall n \in \N \qquad n^2~\text{odd} \quad \Rightarrow \quad n~\text{odd}
	\end{align*}
\end{tm} 
\emph{Flawed proof.} If $n$ is odd then $n = 2k+1$ for some $k \in \N$ and then 
\begin{align*}
n^2 & = (2k+1)^2 \\
 & = 4k^2 + 4k +1 \\
 & = 2 \left( 2k^2 +2k \right) +1
\end{align*}
So $n^2$ is odd.
\begin{proof}
We need to take the following statement for granted, which will be proven later in the document:
\begin{align*}
\forall n \in \N\qquad  \exists k \in \N: ~ n= 2k \quad \vee \quad \exists k \in \N: ~ n = 2k+1
\end{align*}
Assuming that, we argue by contradiction:
\begin{align*}
n \operatorname{even} \quad & \Rightarrow \quad n^2 \operatorname{even} \\
n=2k \quad & \Rightarrow \quad n^2 = 2 \left(2k^2 \right)
\end{align*}
\end{proof}




\subsection{Dis-Proving}
The negation $\overline{P}$ of a statement $P$ can be formed with the help of the two following rules:
\begin{enumerate}
	\item 
	\begin{itemal}
	P & = ( \forall x \in A \quad Q(x)) \\
	\Rightarrow \qquad \overline P & = \left( \exists x \in A\quad \overline{Q(x)} \right)
	\end{itemal}
	\item
	\begin{itemal}
	P & = ( \exists x \in A \quad Q(x) ) \\
	\Rightarrow \qquad \overline P & = \left(\forall x \in A \quad \overline{Q(x)} \right)
	\end{itemal}
\end{enumerate}

\begin{ec}
Show that Rule 2 is the same as Rule 1.
\end{ec}
\begin{rk}
An element $a \in A$ such that $\overline{Q(a)}$ is called a counterexample to the statement
\begin{align*}
(\forall x \in A, \quad Q(x))
\end{align*}
Indeed the very existence of this example $a \in A$ shows that $P$ is false (it "counters" P).
\end{rk}
A typical exam question is:
\begin{qu}
	Prove or disprove the following statement: 
	
	If $p\in \mathbb N$ is prime then $\exists a,b \in \mathbb Z: ~p = a^2+b^2$
\end{qu}


\begin{an}
This statement is false. Counterexample: 3
\begin{align*}
P & = \left(\forall p \in \{n \in \mathbb N|~ n \operatorname{prime}\}: \quad \left(\exists (a,b) \in \mathbb Z^2: p = a^2+b^2\right)\right) \\
\Leftrightarrow \qquad \overline P & = \left(\exists p \in \{n \in \mathbb N|~ n \operatorname{prime}\}: \quad \overline{\left(\exists (a,b) \in \mathbb Z^2: p = a^2+b^2\right)} \right) \\
\Leftrightarrow \qquad \overline P & = \left(\exists p \in \{n \in \mathbb N|~ n \operatorname{prime}\}: \quad \left(\forall (a,b) \in \mathbb Z^2: p \neq a^2+b^2 \right) \right)
\end{align*}
We prove $\overline P$ thus we have to name a particular prime. We choose $p=3$ and claim
\begin{align*}
\forall a,b \in \mathbb Z: \quad a^2+b^2 \neq 3.
\end{align*}
Suppose for contradiction that for some $a,b\in \mathbb Z$, $a^2+b^2=3$. Note that $a^2,b^3 \geq 0$ so both $a^2,b^2 \leq 3$ This means that $|a|,|b| \leq 1$ but then $a^2,b^2 \leq 1$ and $a^2+b^2 \leq 2$. 
\end{an}
